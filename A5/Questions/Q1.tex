\section{}
% Air enters a normal shock at 26 kPa, 230 K, and 815 m/s. Calculate the 
% stagnation pressure and Mach number upstream of the shock, as well as pressure, 
% temperature, velocity, Mach number, and stagnation pressure downstream of the 
% shock. 

\textit{Air enters a normal shock at 26 kPa, 230 K, and 815 m/s. Calculate the stagnation pressure and Mach number upstream of the shock, as well as pressure, temperature, velocity, Mach number, and stagnation pressure downstream of the shock.}

\textbf{Solution}
\begin{itemize}
    \item The flow is steady, adiabatic, and one dimensional
    \item Air is an ideal gas 
\end{itemize}
First, let us calculate the upstream parameters. At room temperature, 
\begin{align*}
    R &= 0.287 \text{ kJ/kg K} \\
    c_p &= 1.005 \text{ kJ/kg K} \\
    k &= 1.4
\end{align*}
Then stagnation temperature is given by
\begin{align*}
    T_0 &= T + \frac{V^2}{2c_p} \\
    &= 230 + \frac{815^2}{2 \cdot 1.005 \cdot 1000} \\
    &= 560.460199 \text{ K}
\end{align*}
Then stagnation pressure is given by
\begin{align*}
    P_0 &= P \left( \frac{T_0}{T} \right)^{k/(k-1)} \\
    &= 26 \left( \frac{560.460199}{230} \right)^{1.4/0.4} \\
    &= \boxed{587.263 \text{ kPa}}
\end{align*}
And Mach number is given by
\begin{align*}
    Ma &= \frac{V}{c} \\
    &= \frac{V}{\sqrt{kR T}} \\
    &= \frac{815}{\sqrt{1.4 \cdot 0.287 \cdot 230 \cdot 1000}} \\
    &= \boxed{2.68095}
\end{align*}
Using Table A14 from Cengel and Cimbala, we can interpolate to calculate the fluid properties \cite{cengel_fluid_2018}.
\begin{table}[H]
    \centering
    \caption{Table A14}
    \begin{tabular}{cccccc}
        \toprule
        Ma$_1$ & Ma$_2$ & $P_2/P_1$ & $T_2/T_1$ & $P_{02}/P_{01}$ \\
        \midrule
        2.6 & 0.5039 & 7.72 & 2.2383 & 0.499 \\
        2.7 & 0.4956 & 8.3383 & 2.3429 & 0.4601 \\
        \midrule
        2.68095 & 0.497181145 & 8.220514236 & 2.322973765 & 0.467510426 \\
        \bottomrule
    \end{tabular}
\end{table}
Where, the last row was calculated using linear interpolation:
\begin{align*}
    y &= \frac{y_i + (y_{i+1} - y_i) \cdot (x - x_i)}{x_{i+1}} 
\end{align*}
Then downstream of the shock, we have
\begin{align*}
    P_2 &= 26 \cdot 8.2205 \\
    &= \boxed{213.7 \text{ kPa}} \\
    T_2 &= 230 \cdot 2.3229 \\ 
    &= \boxed{534.3 \text{ K}} \\
    P_{02} &= 587.263 \cdot 0.4675 \\
    &= \boxed{274.6 \text{ kPa}} 
\end{align*}
For velocity, since $\text{Ma}_2 = 0.4972$, we have
\begin{align*}
    \text{Ma}_2 &= \frac{V_2}{c_2} \\
    \implies V_2 &= \sqrt{kRT_2} \cdot \text{Ma}_2 \\
    &= \sqrt{1.4 \cdot 0.287 \cdot 534.3 \cdot 1000} \cdot 0.4972 \\
    &= \boxed{230.4 \text{ m/s}}
\end{align*}


    
