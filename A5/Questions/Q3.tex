\section{}

\textit{Air flowing at 40 kPa, 210 K, and a Mach number of 3.4 impinges on a two-dimensional wedge of half-angle 8°. Determine the two possible oblique shock angles, $\beta_{\text{weak}}$ and $\beta_{\text{strong}}$, that could be formed by this wedge. For each case, calculate the pressure and Mach number downstream of the oblique shock.}

\textbf{Solution}

Assume
\begin{itemize}
    \item The flow is steady, adiabatic, and one dimensional
    \item Air is an ideal gas 
\end{itemize}
Air at room temperature,
\begin{align*}
    R &= 0.287 \text{ kJ/kg K} \\
    c_p &= 1.005 \text{ kJ/kg K} \\
    k &= 1.4
\end{align*}
The $\theta$-$\beta$-$Ma$ relationship is given by
\begin{align*}
    \tan \theta &=  \frac{2 \cot (\beta) \left(Ma^2 \sin^2 \beta - 1 \right)}{Ma^2 \left(k + \cos(2 \beta) \right) + 2} \\
    \tan (8^\circ) &= \frac{2 \cot (\beta) \left(3.4^2 \sin^2 \beta - 1 \right)}{3.4^2 \left(1.4 + \cos(2 \beta) \right) + 2} \\
    0.140540834702 &= \frac{2 \cot (\beta) \left(11.56 \sin^2 \beta - 1 \right)}{11.56 \left(1.4 + \cos(2 \beta) \right) + 2}
\end{align*}
Then, using MATLAB, (or an approved engineering calculator)
\begin{verbatim}
syms beta positive
eqn = 2 * cotd(beta) * (11.56 * sind(beta)^2 - 1) / (11.56 * (1.4 + cosd(2 * beta)) + 2) ...
== 0.140540834702;
beta1 = vpasolve(eqn, beta, [0, 45]);
beta2 = vpasolve(eqn, beta, [45, 90]);
beta1 = double(beta1)
beta2 = double(beta2)

>> beta1 =

   23.1466


>> beta2 =

   87.4532

\end{verbatim}
Therefore,
\begin{empheq}[box=\fbox]{align*}
    \beta_{\text{weak}} &= 23.1466^\circ \\
    \beta_{\text{strong}} &= 87.4532^\circ
\end{empheq}
Calculating the normal Mach number,
\begin{align*}
    \text{Ma}_{1, n, w} &= \text{Ma}_1 \sin (\beta_{\text{weak}}) \\
    &= 3.4 \sin (23.1466) \\
    &= 1.33648933705 \\
    \text{Ma}_{1, n, s} &= \text{Ma}_1 \sin (\beta_{\text{strong}}) \\
    &= 3.4 \sin (87.4532) \\
    &= 3.39664168189
\end{align*}
Then for weak shock, 
\begin{align*}
    \frac{P_{2, w}}{P_1} &= \frac{2 k \text{Ma}_{1, n, w}^2 - k + 1}{k + 1} \\
    \implies P_{2, w} &= 40 \cdot \frac{2 \cdot 1.4 \cdot (1.33648933705)^2 - 1.4 + 1}{1.4 + 1} \\
    &= \boxed{76.69 \text{ kPa}}
\end{align*}
And for strong shock,
\begin{align*}
    \frac{P_{2, s}}{P_1} &= \frac{2 k \text{Ma}_{1, n, s}^2 - k + 1}{k + 1} \\
    \implies P_{2, s} &= 40 \cdot \frac{2 \cdot 1.4 \cdot (3.39664168189)^2 - 1.4 + 1}{1.4 + 1} \\
    &= \boxed{531.7 \text{ kPa}}
\end{align*}
The normal Mach number downstream of the weak shock is given by
\begin{align*}
    \text{Ma}_{2, n, w} &= \sqrt{\frac{(k - 1) \text{Ma}_{1, n, w}^2 + 2}{2k \text{Ma}_{1, n, w}^2 - k + 1}} \\
    &= \sqrt{\frac{(1.4 - 1) (1.33648933705)^2 + 2}{2 \cdot 1.4 \cdot (1.33648933705)^2 - 1.4 + 1}} \\
    &= 0.768068300414
\end{align*}
And for the strong shock,
\begin{align*}
    \text{Ma}_{2, n, s} &= \sqrt{\frac{(k - 1) \text{Ma}_{1, n, s}^2 + 2}{2k \text{Ma}_{1, n, s}^2 - k + 1}} \\
    &= \sqrt{\frac{(1.4 - 1) (3.39664168189)^2 + 2}{2 \cdot 1.4 \cdot (3.39664168189)^2 - 1.4 + 1}} \\
    &= 0.455341755517
\end{align*}
Therefore, the downstream Mach numbers for the weak shock is 
\begin{align*}
    \text{Ma}_{2, w} &= \frac{\text{Ma}_{2, n, w}}{\sin (\beta_{\text{weak}} - \theta)} \\
    &= \frac{0.768068300414}{\sin (23.1466 - 8)} \\
    &= \boxed{2.940}
\end{align*}
And for the strong shock,
\begin{align*}
    \text{Ma}_{2, s} &= \frac{\text{Ma}_{2, n, s}}{\sin (\beta_{\text{strong}} - \theta)} \\
    &= \frac{0.455341755517}{\sin (87.4532 - 8)} \\
    &= \boxed{0.4632}
\end{align*}
